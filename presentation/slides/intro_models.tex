\begin{frame}[squeeze]{Główne sposoby modelowania problemu}
    \begin{itemize}
        \setlength\itemsep{0.6em}
        \item Model klasyczny
        \item Model strumieniowy 
        \item Model półstrumieniowy
        \item Model rozproszony
    \end{itemize}

    \begin{definicja}[Strumień grafowy]
        Strumieniem grafowym nazywamy ciągłą sekwencję elementów, z których każdy ma postać trójki: 
        \[
            e_i = (<s_i, d_i>; w_i, t_i),
        \]
        gdzie $s_i, d_i$ wierzchołki grafu i przez parę $<s_i, d_i>$ oznaczamy krawędź pomiędzy nimi. Z kolei $w_i$ i $t_i$ to odpowiednio waga tej krawędzi i moment jej wystąpienia.
    \end{definicja}
\end{frame}