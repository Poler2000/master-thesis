\begin{frame}[squeeze]{Wpływ rozmiaru szkicu na precyzję rekonstrukcji}

    \begin{table}[t]
        \small
        \centering
        \resizebox{\columnwidth}{!}{%
        \begin{tabular}{|l|l|l|l|l|l|l|l|}
        \hline
        & & \multicolumn{3}{c|}{NodeSketch} & \multicolumn{3}{c|}{EdgeSketch} \\ \cline{1-8}
                \textbf{p} & \textbf{k} & \textbf{t = 1000} & \textbf{t = 10000} & \textbf{t = |E|} & \textbf{t = 1000} & \textbf{t = 10000} & \textbf{t = |E|} \\ \hline\hline
                \multirow{3}{*}{0.001} & 2 & 0 & 0.0016 & 0.0055 & 1 & 1 & 0.3721 \\ \cline{2-8}
                & 3 & 0.017 & 0.012 & 0.006 & 1 & 0.9671 & 0.4235 \\ \cline{2-8}
                & 4 & 0.003 & 0.0084 & 0.0099 & 1 & 0.9714 & 0.4275 \\ \hline
        \end{tabular}}
        \caption{Precyzja uzyskiwana przez algorytmy NodeSketch i EdgeSketch dla grafu ważonego w modelu Erdosa-Renyiego. Parametry grafu:
        \begin{itemize}
            \item $10000$ wierzchołków.
            \item Prawdopodobieństwo krawędzi $p = 0.001$.
            \item Średni stopień wierzchołka ok. $10$.
            \item Rozmiar szkicu $m = 10$.
        \end{itemize}}

        \label{tab:weighted_graphs}
    \end{table}
\end{frame}
