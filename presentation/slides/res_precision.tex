\begin{frame}[squeeze]{Wpływ rozmiaru szkicu na precyzję rekonstrukcji}

    \begin{table}[t]
        \small
        \centering
        \resizebox{\columnwidth}{!}{%
        \begin{tabular}{|l|l|l|l|l|l|l|l|l|l|}
            \hline
            & & \multicolumn{3}{c|}{NodeSketch} & \multicolumn{3}{c|}{EdgeSketch} \\ \cline{1-8}
            \textbf{$b$} & \textbf{k} & \textbf{t = 1000} & \textbf{t = 10000} & \textbf{t = |E|} & \textbf{t = 1000} & \textbf{t = 10000} & \textbf{t = |E|} \\ \hline\hline
            \multirow{3}{*}{4} & 2 & 0.542 & 0.5343 & 0.5131 & 1 & 0.3352 & 0.1547 \\ \cline{2-8}
             & 3 & 0.483 & 0.4991 & 0.5003 & 1 & 0.5342 & 0.4087 \\ \cline{2-8}
             & 4 & 0.499 & 0.4983 & 0.5008 & 1 & 0.5342 & 0.4087 \\ \hline\hline
        \end{tabular}}
        \caption{Precyzja uzyskiwana przez algorytmy NodeSketch i EdgeSketch dla grafu w stochastycznym modelu blokowym w zależności od wielkości próbek $t$. Parametry grafu: \\
        \begin{itemize}
            \item $1000$ wierzchołków podzielonych na $b = 4$ bloki.
            \item Prawdopodobieństwo krawędzi wewnątrz bloku $p = 0.5$ i pomiędzy blokami $q = 0.001$.
            \item Średni stopień wierzchołka ok. $125$.
            \item Rozmiar szkicu $m = 10$.
        \end{itemize}}
        \label{tab:stochastic_block_model}
    \end{table}
\end{frame}