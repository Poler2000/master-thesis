\chapter{Analiza wyników}

    W niniejszym rozdziale zweryfikowano eksperymentalnie działanie algorytmu EdgeSketch oraz porównano go z algorytmem NodeSketch. Przeprowadzono szereg testów na różnych grafach i zbadano wpływ parametrów na jakość uzyskiwanych wyników. Sprawdzono także średnią liczbę operacji wykonywanych przez algorytmy w zależności od liczby krawędzi w grafie.

\section{Architektura eksperymentów}

    Na działanie algorytmów NodeSketch i EdgeSketch wpływ mają trzy główne parametry. Są to:
    \begin{itemize}
        \item $k$ - rząd sąsiedztwa. Przyjmujemy, że przy $k = 2$ rozpatrujemy tylko bezpośrednie połączenia, przy $k = 3$ także ścieżki długości $2$ itd. 
        \item $m$ - rozmiar szkicu. Jest to liczba elementów w wektorze opisującym pojedynczy wierzchołek w grafie. 
        \item $\alpha$ - parametr rozkładu wykładniczego. Decyduje on, jakie wagi nadawane są sąsiedztwom wyższych rzędów.
    \end{itemize}

    Jednak uzyskiwane wyniki zależą też w naturalny sposób od grafu, na którym przeprowadzane są eksperymenty. W związku z tym, w celu zbadania wpływu tych parametrów na jakość uzyskiwanych wyników, przeprowadzono szereg eksperymentów na różnych grafach. Główną badaną statystyką była precyzja, czyli stosunek liczby poprawnie odgadniętych krawędzi do wartości $t$, a więc wielkości próbki. Rozważano próbki wielkości $t \in \{100, 1000, 10000, |E|\}$, gdzie $|E|$ to liczba krawędzi w grafie. Algorytmy zostały zaimplementowane w języku Julia\cite{Julia}.

\section{Badanie uzyskiwanych wyników w zależności od struktury grafów}

    \subsection{Wyniki}
    \subsubsection{Model Erdosa-Renyiego}
        Model Erdosa-Renyiego jednym z najpowszechniejszych modeli grafów losowych. W modelu tym, każda para wierzchołków jest połączona krawędzią z jednakowym prawdopodobieństwem $p$. Steruje on gęstością grafu, a co za tym idzie, średnim stopniem wierzchołków. Grafy w tym modelu charakteryzują się dość jednorodną strukturą, bez wyraźnych klastrów, czy wierzchołków o dysproporcjonalnie wysokich stopniach. 

        Eksperyment przeprowadzono dla różnych prawdopodobieństw wystąpienia krawędzi $p$ oraz stopni sąsiedztwa $k \in \{2,3,4\}$. W każdym przypadku liczba wierzchołków w grafie wynosiła $10000$. Rozmiar szkicu wyniósł $m = 10$, a parametr rozkładu wykładniczego $\alpha = 0.3$. Wyniki przedstawiono w tabeli \ref{tab:erdos_renyi}.

        \begin{table}[!ht]
        \small
            \centering
            \begin{tabular}{|l|l|l|l|l|l|l|l|l|l|}
            \hline
                & & \multicolumn{4}{c|}{NodeSketch} & \multicolumn{4}{c|}{EdgeSketch} \\ \cline{1-10}
                \textbf{p} & \textbf{k} & \textbf{t = 100} & \textbf{t = 1000} & \textbf{t = 10000} & \textbf{t = |E|} & \textbf{t = 100} & \textbf{t = 1000} & \textbf{t = 10000} & \textbf{t = |E|} \\ \hline\hline
                \multirow{3}{*}{0.0005} & 2 & 0.94 & 0.762 & 0.4745 & 0.3688 & 1	& 1	& 1	& 0.5562 \\ \cline{2-10}
                & 3 & 0.86 & 0.732 & 0.3947 & 0.2515	& 1	& 1	& 0.9446 &	0.6179 \\ \cline{2-10}
                & 4 & 0.55 & 0.164 & 0.0409 & 0.023	& 1	& 0.985 & 0.8983 & 0.6072 \\ \hline\hline
                \multirow{3}{*}{0.001} & 2 & 0.58 & 0.433 & 0.2779 & 0.2048	& 1	& 1	& 1	& 0.3721 \\ \cline{2-10}
                & 3 & 0.38 & 0.074 & 0.0185 & 0.007	& 1	& 1	& 0.9748 & 0.4285 \\ \cline{2-10}
                & 4 & 0 & 0.004 & 0.0012 & 0.0017 & 1 & 1 & 0.9665 & 0.4301 \\ \hline\hline
                \multirow{3}{*}{0.005} & 2 & 0.26 & 0.144 & 0.0901 & 0.0497	& 1	& 1	& 1	& 0.0981 \\ \cline{2-10}
                & 3 & 0.11 & 0.033	& 0.0137 & 0.0062 & 1 & 0.936 & 0.3025 & 0.0557 \\ \cline{2-10}
                & 4 & 0.01 & 0.006 & 0.0038 & 0.005	& 1	& 0.936	& 0.3025 & 0.0557 \\ \hline\hline
                \multirow{3}{*}{0.01} & 2 & 0.05 & 0.079 & 0.0595 & 0.033	& 1	& 1	& 0.9998 & 0.0576 \\ \cline{2-10}
                & 3 & 0.05 & 0.026 & 0.0157 & 0.0104	& 1	& 1	& 1	& 0.05 \\ \cline{2-10}
                & 4 & 0 & 0.006 & 0.0091 & 0.0102	& 1	& 1	& 1	& 0.05 \\ \hline
            \end{tabular}
            \label{tab:erdos_renyi}
            \caption{Model Erdosa-Renyiego}

        \end{table}

    \subsubsection{Stochastyczny model blokowy}
    Stochastyczny model blokowy (\emph{ang. stochastic block model}) jest modelem, w którym wierzchołki grafu są podzielone losowo na $b$ bloków. Prawdopodobieństwo wystąpienia krawędzi między dwoma wierzchołkami w tym samym bloku wynosi $p$, a pomiędzy wierzchołkami z różnych bloków $q$. Zazwyczaj przyjmuje się $p \gg q$. W modelu tym występują wyraźne klastry, ale stopnie poszczególnych wierzchołków są do siebie dość zbliżone. Sprawdza się on w symulowaniu zbiorów danych, w których można wyróżnić grupy podobnych do siebie punktów, jak np. grupy użytkowników o podobnych zainteresowaniach, albo artykuły z podobnymi słowami kluczowymi. 

    W przeprowadzonym eksprymencie przyjęto rozmiar grafu $n = 1000$, prawdopodobieństwo wystąpienia krawędzi wewnątrz bloku $p = 0.5$ oraz między blokami $q = 0.001$. Zbadano także różne liczby bloków $b \in \{2,4,8\}$. Rozmiar szkicu $m$ ustalono na $10$, a paramter rozkładu wykładniczego wyniósł $\alpha = 0.3$. Wyniki przedstawiono w tabeli \ref{tab:stochastic_block_model}.

    \begin{table}[!ht]
        \small
        \centering
        \begin{tabular}{|l|l|l|l|l|l|l|l|l|l|}
        \hline
            & & \multicolumn{4}{c|}{NodeSketch} & \multicolumn{4}{c|}{EdgeSketch} \\ \cline{1-10}
            \textbf{bloki} & \textbf{k} & \textbf{t = 100} & \textbf{t = 1000} & \textbf{t = 10000} & \textbf{t = |E|} & \textbf{t = 100} & \textbf{t = 1000} & \textbf{t = 10000} & \textbf{t = |E|} \\ \hline\hline
            \multirow{3}{*}{2} & 2 & 0.57 & 0.525 & 0.5136 & 0.5072 & 1 & 1 & 0.4391 & 0.2616 \\ \cline{2-10}
             & 3 & 0.47 & 0.527 & 0.5089 & 0.5016 & 1 & 1 & 0.4545 & 0.3426 \\ \cline{2-10}
             & 4 & 0.44 & 0.523 & 0.5079 & 0.5022 & 1 & 1 & 0.4545 & 0.3426 \\ \hline\hline
            \multirow{3}{*}{4} & 2 & 0.5 & 0.542 & 0.5343 & 0.5131 & 1 & 1 & 0.3352 & 0.1547 \\ \cline{2-10}
             & 3 & 0.53 & 0.483 & 0.4991 & 0.5003 & 1 & 1 & 0.5342 & 0.4087 \\ \cline{2-10}
             & 4 & 0.46 & 0.499 & 0.4983 & 0.5008 & 1 & 1 & 0.5342 & 0.4087 \\ \hline\hline
            \multirow{3}{*}{8} & 2 & 0.66 & 0.594 & 0.5521 & 0.5234 & 1 & 1 & 0.2831 & 0.1289 \\ \cline{2-10}
             & 3 & 0.51 & 0.528 & 0.5158 & 0.5063 & 1 & 0.884 & 0.5825 & 0.4762 \\ \cline{2-10}
             & 4 & 0.5 & 0.496 & 0.5002 & 0.5029 & 1 & 0.884 & 0.5821 & 0.4755 \\ \hline
        \end{tabular}
        \label{tab:stochastic_block_model}
        \caption{Stochastyczny model blokowy}
    \end{table}

    \subsubsection{Model Barabasiego-Alberta}
    W modelu Barabasiego-Alberta (\emph{ang. Barabasi-Albert Model}) wierzchołki dodawane są do grafu sekwencyjnie. Nowy wierzchołek łączy się z istniejącymi wierzchołkami z prawdopodobieństwem proporcjonalnym do ich stopnia. Bardziej formalnie, konstrukcja grafu rozpoczyna się od wyboru małego zbioru $m_0$ początkowycyh wierzchołków i poprowadzenia wśród nich krawędzi w taki sposób, aby każdy wierzchołek był połączony z co najmniej jednym innym. Następnie, w każdym kroku dodawany jest nowy wierzchołek wraz z $m_{ba}$ krawędziami łączącymi go z innymi z prawdopodobieństwem tym większym, im większy jest stopień danego wierzchołka. Konkretnie, prawdopodobieństwo istnienia krawędzi do wierzchołka $i$ wynosi 
    \[
        p_i = \frac{k_i}{\sum_{j \in V^{*}} k_j},
    \]
    gdzie $k_i$ to stopień wierzchołka $i$, a $V^{*}$ to zbiór aktualnie występujących w grafie wierzchołków. W powstałych w ten sposób grafach można zaobserwować duże zróżnicowanie stopni wierzchołków, z wyraźnie odznaczającymi się węzłami centralnymi (\emph{ang. hubs}).

    W przeprowadzonym eksperymencie przyjęto liczbę wierzchołków $n = 1000$ i stopień wierzchołków $m_{ba} \in \{2,4,8\}$, a rozmiar początkowego zbioru wierzchołków ustalono na $m_0 = m_{ba}$. Wyniki przedstawiono w tabeli \ref{tab:barabasi_albert}. W przypadku $m_{ba} = 2$ pominięto wyniki dla $t = 10000$, ponieważ liczba wierzchołków w grafie wynosiła tylko $3994$.
    \begin{table}[!ht]
        \small
        \centering
        \begin{tabular}{|l|l|l|l|l|l|l|l|l|l|}
        \hline
            & & \multicolumn{4}{c|}{NodeSketch} & \multicolumn{4}{c|}{EdgeSketch} \\ \cline{1-10}
            \textbf{m} & \textbf{k} & \textbf{t = 100} & \textbf{t = 1000} & \textbf{t = 10000} & \textbf{t = |E|} & \textbf{t = 100} & \textbf{t = 1000} & \textbf{t = 10000} & \textbf{t = |E|} \\ \hline\hline
            \multirow{3}{*}{2} & 2 & 0.52 & 0.269 & X & 0.2163 & 1 & 0.974 & X & 0.4877 \\ \cline{2-10}
            & 3 & 0.32 & 0.152 & X & 0.1232 & 1 & 0.213 & X & 0.1567 \\ \cline{2-10}
            & 4 & 0.56 & 0.289 & X & 0.2153 & 1 & 0.134 & X & 0.0941 \\ \hline\hline
            \multirow{3}{*}{8} & 2 & 0.19 & 0.14 & 0.0709 & 0.0892 & 1 & 1 & 0.178 & 0.2241 \\ \cline{2-10}
            & 3 & 0.12 & 0.063 & 0.0406 & 0.0511 & 1 & 0.921 & 0.1765 & 0.2222 \\ \cline{2-10}
            & 4 & 0.08 & 0.07 & 0.0313 & 0.0394 & 1 & 0.921 & 0.1765 & 0.2222 \\ \hline\hline
            \multirow{3}{*}{16} & 2 & 0.12 & 0.079 & 0.0939 & 0.0939 & 1 & 1 & 0.2105 & 0.1424 \\ \cline{2-10}
            & 3 & 0.1 & 0.107 & 0.0656 & 0.0595 & 1 & 1 & 0.2197 & 0.1578 \\ \cline{2-10}
            & 4 & 0.06 & 0.028 & 0.0359 & 0.0376 & 1 & 1 & 0.2197 & 0.1578 \\ \hline
        \end{tabular}
        \label{tab:barabasi_albert}
        \caption{Model Barabasiego-Alberta}
    \end{table}

    \subsubsection{Grafy ważone}

    Wyniki dla grafów ważonych przedstawiono w tabeli \ref{tab:weighted_graphs}.

    \begin{table}[!ht]
        \small
        \centering
        \begin{tabular}{|l|l|l|l|l|l|l|l|l|l|}
        \hline
        & & \multicolumn{4}{c|}{NodeSketch} & \multicolumn{4}{c|}{EdgeSketch} \\ \cline{1-10}
                \textbf{p} & \textbf{k} & \textbf{t = 100} & \textbf{t = 1000} & \textbf{t = 10000} & \textbf{t = |E|} & \textbf{t = 100} & \textbf{t = 1000} & \textbf{t = 10000} & \textbf{t = |E|} \\ \hline\hline
            \multirow{3}{*}{0.0005} & 2 & 0 & 0 & 0.0012 & 0.0031 & 1 & 1 & 1 & 0.5505 \\ \cline{2-10}
            & 3 & 0 & 0 & 0.007 & 0.0248 & 1 & 1 & 0.94 & 0.6125 \\ \cline{2-10}
            & 4 & 0 & 0.006 & 0.0122 & 0.012 & 1 & 1 & 0.9207 & 0.6073 \\ \hline\hline
            \multirow{3}{*}{0.001} & 2 & 0 & 0 & 0.0016 & 0.0055 & 1 & 1 & 1 & 0.3721 \\ \cline{2-10}
            & 3 & 0.01 & 0.017 & 0.012 & 0.006 & 1 & 1 & 0.9671 & 0.4235 \\ \cline{2-10}
            & 4 & 0 & 0.003 & 0.0084 & 0.0099 & 1 & 1 & 0.9714 & 0.4275 \\ \hline\hline
            \multirow{3}{*}{0.005} & 2 & 0.01 & 0.008 & 0.0054 & 0.0065 & 1 & 1 & 1 & 0.0989 \\ \cline{2-10}
            & 3 & 0 & 0.007 & 0.0086 & 0.0063 & 1 & 0.941 & 0.3344 & 0.0581 \\ \cline{2-10}
            & 4 & 0 & 0.004 & 0.0064 & 0.0088 & 1 & 0.941 & 0.3344 & 0.0581 \\ \hline\hline
            \multirow{3}{*}{0.01} & 2 & 0 & 0.01 & 0.0086 & 0.0107 & 1 & 1 & 1 & 0.0583 \\ \cline{2-10}
            & 3 & 0.03 & 0.012 & 0.0101 & 0.0117 & 1 & 1 & 1 & 0.05 \\ \cline{2-10}
            & 4 & 0.01 & 0.011 & 0.0151 & 0.0138 & 1 & 1 & 1 & 0.05 \\ \hline
        \end{tabular}
        \label{tab:weighted_graphs}
        \caption{Grafy ważone}
    \end{table}

    \subsection{Wnioski}

\section{Wykorzystanie stopnia wierzchołków przy rekonstrukcji}

    \subsection{Wyniki}

    Wyniki przedstawiono w tabeli \ref{tab:degree}.

    \begin{table}[!ht]
        \small
        \centering
        \begin{tabular}{|l|l|l|l|l|l|l|l|l|l|}
        \hline
        & & \multicolumn{4}{c|}{EdgeSketch} & \multicolumn{4}{c|}{EdgeSketch z oszacowaniem stopnia} \\ \cline{1-10}
            \textbf{zbiór danych} & \textbf{k} & \textbf{t = 100} & \textbf{t = 1000} & \textbf{t = 10000} & \textbf{t = |E|} & \textbf{t = 100} & \textbf{t = 1000} & \textbf{t = 10000} & \textbf{t = |E|} \\ \hline\hline
            \multirow{3}{*}{HomoSapiens} & 2 & 1 & 1 & 0.4459 & 0.1202 & 0.99 & 0.998 & 0.4413 & 0.1669 \\ \cline{2-10}
            & 3 & 1 & 1 & 0.3036 & 0.1153 & 1 & 0.999 & 0.2923 & 0.1576 \\ \cline{2-10}
            & 4 & 1 & 1 & 0.3035 & 0.1163 & 1 & 1 & 0.2921 & 0.1573 \\ \hline\hline
            \multirow{3}{*}{blogcatalog} & 2 & 1 & 1 & 0.7757 & 0.0343 & 1 & 1 & 0.7755 & 0.1943 \\ \cline{2-10}
            & 3 & 1 & 1 & 0.7649 & 0.0377 & 1 & 1 & 0.7647 & 0.1988 \\ \cline{2-10}
            & 4 & 1 & 1 & 0.7649 & 0.0377 & 1 & 1 & 0.7647 & 0.1988 \\ \hline\hline
            \multirow{3}{*}{dblp} & 2 & 1 & 1 & 1 & 0.4453 & 0.99 & 0.994 & 0.9489 & 0.426 \\ \cline{2-10}
            & 3 & 1 & 1 & 0.7684 & 0.3214 & 1 & 0.995 & 0.731 & 0.449 \\ \cline{2-10}
            & 4 & 1 & 1 & 0.6666 & 0.2927 & 1 & 0.997 & 0.6317 & 0.4303 \\ \hline
        \end{tabular}
        \label{tab:degree}
        \caption{Wyniki z wykorzystaniem stopnia wierzchołków przy rekonstrukcji}
    \end{table}

    \subsection{Wnioski}

\section{Wpływ rozmiaru szkicu na uzyskiwane wyniki}

    Pierwszy eksperyment przeprowadzono na zbiorach danych blogcatalog oraz dblp. W obu przypadkach rozważano różne wartości parametru $m \in \{8, 16, 32, 64, 128, 256\}$. Wyniki przedstawiono w tabeli \ref{tab:sketch_size}.

    \subsection{Wyniki}
    \begin{table}[!ht]
        \small
        \centering
        \begin{tabular}{|l|l|l|l|l|l|l|l|}
        \hline
        & & \multicolumn{3}{c|}{NodeSketch} & \multicolumn{3}{c|}{EdgeSketch} \\ \cline{1-8}
                \textbf{zbiór danych} & \textbf{m} & \textbf{t = 1000} & \textbf{t = 10000} & \textbf{t = |E|} & \textbf{t = 1000} & \textbf{t = 10000} & \textbf{t = |E|} \\ \hline\hline
        \multirow{6}{*}{blogcatalog} & 8 & 0.031 & 0.0298 & 0.0277 & 1 & 0.6228 & 0.0298 \\ \cline{2-8}
            & 16 & 0.023 & 0.0276 & 0.0292 & 1 & 0.9998 & 0.0474 \\ \cline{2-8}
            & 32 & 0.009 & 0.0224 & 0.034 & 1 & 0.9998 & 0.0805 \\ \cline{2-8}
            & 64 & 0.016 & 0.0224 & 0.0312 & 1 & 1 & 0.138 \\ \cline{2-8}
            & 128 & 0.01 & 0.0185 & 0.0466 & 1 & 1 & 0.2302 \\ \cline{2-8}
            & 256 & 0.018 & 0.0257 & 0.0406 & 1 & 1 & 0.3609 \\ \hline\hline
        \multirow{6}{*}{dblp} & 8 & 0.996 & 0.8975 & 0.5495 & 1 & 1 & 0.3839 \\ \cline{2-8}
            & 16 & 1 & 0.9381 & 0.5892 & 1 & 1 & 0.5737 \\ \cline{2-8}
            & 32 & 1 & 0.9264 & 0.5974 & 1 & 1 & 0.7481 \\ \cline{2-8}
            & 64 & 1 & 0.9333 & 0.589 & 1 & 1 & 0.87 \\ \cline{2-8}
            & 128 & 1 & 0.9389 & 0.6017 & 1 & 1 & 0.9421 \\ \cline{2-8}
            & 256 & 1 & 0.9415 & 0.6051 & 1 & 1 & 0.9757 \\ \hline
        \end{tabular}
        \label{tab:sketch_size}
        \caption{Wyniki dla różnych rozmiarów szkicu}
    \end{table}
    \subsection{Wnioski}
    Zgodnie z intuicją, zwiększenie rozmiaru szkicu ma korzystny wpływ na uzyskiwaną precyzję. Algorytm EdgeSketch jest szczególnie widoczne w przypadku dużych wartości $t$. Warto jednak zaznaczyć, że im większy szkic, tym dalsze jego zwiększanie ma mniejszy wpływ na precyzję. Co ciekawe, dla małych $t$, EdgeSketch jest w stanie uzyskać wysoką precyzję nawet przy małym rozmiarze szkicu.  W przypadku algorytmu NodeSketch, zyski występują, ale są niewielki w porównaniu do EdgeSketcha. Warto zaznaczyć, że większy rozmiar szkicu implikuje większą liczbę wykonywanych operacji oraz większe zapotrzebowanie na pamięć, dlatego w praktycznych zastosowaniach konieczne może być znaleznienie kompromisu pomiędzy jakością uzyskiwanych wyników a kosztem obliczeniowym.

\section{Dobór parametru alpha}

    \subsection{Wyniki}

    \subsection{Wnioski}

\section{Liczba operacji w zależności od liczby krawędzi w grafie}
\label{sec:performance}