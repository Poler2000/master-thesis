\chapter{Analiza wyników}

    W niniejszym rozdziale 

\section{Architektura eksperymentów}

    Na działanie algorytmów NodeSketch i EdgeSketch wpływ mają trzy główne parametry. Są to:
    \begin{itemize}
        \item $k$ - rząd sąsiedztwa. Przyjmujemy, że przy $k = 2$ rozpatrujemy tylko bezpośrednie połączenia, przy $k = 3$ także ścieżki długości $2$ itd. Zależy od niego liczba rekurencyjnych wywołań algorytmu.
        \item $m$ - rozmiar szkicu. Jest to liczba , służących do opisu pojedynczego wierzchołka w grafie. 
        \item $\alpha$ - . Parametr ten decyduje, jakie wagi nadawane są sąsiedztwom wyższych rzędów
    \end{itemize}

    . Jednak uzyskiwane wyniki zależą też w naturalny sposób od grafu, na którym przeprowadzane są eksperymenty. W związku z tym, w celu zbadania wpływu tych parametrów na jakość uzyskiwanych wyników, przeprowadzono szereg eksperymentów na różnych grafach. Algorytmy zostały zaimplementowane w języku Julia.

\section{Badanie uzyskiwanych wyników w zależności od rozmiaru i struktury grafów}
    \subsection{Grafy ważone}
    \subsection{Wyniki}

    \subsection{Wnioski}

\section{Wykorzystanie stopnia wierzchołków przy rekonstrukcji}

    \subsection{Wyniki}

    \subsection{Wnioski}

\section{Wpływ rozmiaru szkicu na uzyskiwane wyniki}

    \subsection{Wyniki}

    \subsection{Wnioski}

\section{Dobór parametru alpha}

    \subsection{Wyniki}

    \subsection{Wnioski}

\section{Liczba operacji w zależności od liczby krawędzi w grafie}
\label{sec:performance}