\chapter{Wstęp}

TODO: Tutaj będzie bardzo ogólne wprowadzenie do pracy.
\begin{itemize}
    \item Motywacja
    \item Krótki i "wysokopoziomowy" opis problemu
    \item Podsumowanie osiągnięć pracy
\end{itemize}

\section{Struktura pracy}
TODO: Kilka(dziesiąt) słów o strukturze i zawartości pracy. Omówienie po kolei rozdziałów i ewentualnych dodatków. Bardzo skrótowo, bo wszystko będziemy potem  i tak rozwijać. Coś w jak niżej (do dopracowania).

Pierwszy rozdział stanowi niniejszy wstęp, przedstawiający ogólny zarys problematyki pracy i skrótowo podsumowujący jej wkład badawczy. W drugim rozdziale znajduje się opis problemu wraz z formalną definicją i przedstawieniem różnych jego wariantów. Przedmiotem trzeciego rozdziału jest przegląd literatury związanej z analizą wielkich grafów, z podziałem na zastosowane metodyki oraz tabelą ilustrującą porównanie znanych struktur i algorytmów.  W czwartym rozdziale szczegółowo omówione zostały szkice danych, od ich formalnej definicji do bardziej praktycznych przykładów ich wykorzystanie, także w kontekście niniejszej pracy. Piąty rozdział zawiera właściwy opis tego, co zostało zrobione [TODO: przepisać nieco bardziej szczegółowo]. W rozdziale szóstym opisane zostały przeprowadzone eksperymenty, wraz z prezentacją wyników oraz wnioskami z nich płynącymi. Ostatni, siódmy rozdział, stanowi podsumowanie pracy. Zawarte zostały w nim ogólne konkluzje na temat pracy oraz możliwe kierunki dalszych badań. Pracy towarzyszy wykaz literatury oraz dodatek, zawierający opis dołączonej płyty CD [TODO: czy nadal załączamy płytę?] i instrukcję użytkowania części implementacyjnej. 