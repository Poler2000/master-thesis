\chapter*{Wstęp}
\addcontentsline{toc}{chapter}{Wstęp}

W dobie internetu i powszechnej informatyzacji zachodzi potrzeba przetwarzania coraz większych zbiorów informacji. Wiele z nich wygodnie jest traktować jako dane grafowe. Stanowi to jednak wyzwanie dla tradycyjnych algorytmów i struktur danych, które często okazują się nieefektywne dla grafów o milionach, a nawet miliardach krawędzi, zarówno pod względem czasu obliczeń, jak i potrzebnych zasobów pamięciowych.
W ostatnich latach temat ten stał się przedmiotem intensywnych badań, czego skutkiem było powstanie wielu efektywnych rozwiązań, pozwalających na analizowanie nawet bardzo dużych grafów w rozsądnym czasie. Wiele z nich wykorzystuje do tego celu szkice danych, czyli kompaktowe reprezentacje oryginalnych danych. W niniejszej pracy przyglądamy się takim rozwiązaniom, a w szczególności algorytmowi \texttt{NodeSketch}, tworzącemu szkice grafów poprzez rekurencyjne podsumowywanie otoczeń wierzchołków i generowanie na ich podstawie próbek z rozkładu wykładniczego. Na jego podstawie definiujemy algorytm \texttt{EdgeSketch}, wykorzystujący metodę \texttt{FastExpSketch} do szkicowania wierzchołków. Modyfikacja ta pozwala na wykonywanie bardziej złożonych operacji na szkicach. Przeprowadzone eksperymentu pokazują, że pomysł ten dobrze sprawdza się w praktyce, oferując lepszą precyzję przy rekonstrukcji grafu w stosunku do oryginalnego algorytmu \texttt{NodeSketch}, a także zmniejszając średnią liczbę drogich operacji, jak np. obliczanie wartości funkcji haszującej.   

\subsection*{Struktura pracy}
Praca rozpoczyna się od niniejszego wstępu, przedstawiającego ogólny zarys podejmowanej problematyki i skrótowo podsumowującego jej wkład badawczy. W pierwszym rozdziale znajduje się opis problemu wraz z formalną definicją i przedstawieniem różnych jego wariantów. Przedmiotem drugiego rozdziału jest przegląd literatury związanej z analizą wielkich grafów, z podziałem na zastosowane metodyki. W trzecim rozdziale szczegółowo omówione zostały algorytmy \texttt{ExpSketch} i \texttt{FastExpSketch}, a także operacje na szkicach danych. Czwarty rozdział zawiera ideę oraz dokładny opis działania algorytmu \texttt{EdgeSketch}, stanowiącego główny przedmiot pracy. W piątym i zarazem ostatnim rozdziale opisane zostały przeprowadzone eksperymenty, wraz z prezentacją wyników oraz wnioskami z nich płynącymi. Następnym elementem jest podsumowanie pracy. Zawarte zostały w nim ogólne konkluzje na temat pracy oraz możliwe kierunki dalszych badań. Pracy towarzyszy wykaz literatury oraz dodatek, zawierający instrukcję użytkowania części implementacyjnej. 
