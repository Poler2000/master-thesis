\chapter*{Podsumowanie}
\addcontentsline{toc}{chapter}{Podsumowanie}
\label{chap:summary}

    W ramach niniejszej pracy zdefiniowano problem analizy wielkich grafów w modelu strumieniowym oraz omówiono praktyczne zastosowania takiego modelu. Dokonano także przeglądu istniejących algorytmów i struktur danych, które znajdują bądź mogą znaleźć swoje zastosowanie w analizie danych grafowych. Szczególną uwagę poświęcono metodom korzystającym ze szkiców danych oraz tworzącym zanurzenia wierzchołków, czyli ich kompaktowe reprezentacje w niskowymiarowych przestrzeniach wektorowych. Jednym z takich algorytmów jest \texttt{NodeSketch}, który szkicuje wierzchołki grafu, rekurencyjnie analizując ich $k$-sąsiedztwo. Stał się on punktem wyjścia do konstrukcji algorytmu \texttt{EdgeSketch}, łączącego główny schemat działania \texttt{NodeSketch} z metodą szkicowania wierzchołków znaną z algorytmu \texttt{FastExpSketch}. Pozwoliło to, między innymi, na efektywne wykonywanie operacji teoriomnogościowych na szkicach. Przeprowadzono teoretyczną analizę złożoności obliczeniowej, która została następnie zweryfikowana eksperymentalnie. Testy na zróżnicowanych zbiorach danych pokazały, że tak skonstruowany algorytm osiąga w większości wypadków lepszą precyzję przy rekonstrukcji krawędzi grafu niż \texttt{NodeSketch}. Wyniki te potwierdzają, że \texttt{EdgeSketch} jest skutecznym w praktyce algorytmem i stanowi wartościowy wkład badawczy w dziedzinę analizy dużych grafów.

\subsubsection*{Potencjalne kierunki rozwoju}
    Badanie technik generowania szkiców danych grafowych, takich jak te zastosowane do opracowania algorytmu \texttt{EdgeSketch}, oraz ich zastosowań jest niezwykle ciekawym i szerokim zagadnieniem, które trudno zamknąć w obrębie jednej pracy. Stąd też istnieje kilka potencjalnych obszarów, które mogą stanowić rozwinięcie niniejszej pracy. 
    
    Jednym z podstawowych kierunków rozwoju jest oczywiście dalsze badanie algorytmu \texttt{EdgeSketch}. Bardziej szczegółowe testy lub analizy teoretyczne mogłyby pozwolić na jeszcze lepsze zrozumienie, jaki wpływ na uzyskiwane wynika ma charakterystyka grafu i pozwolić na optymalizację doboru parametrów.

    Ponadto, rekonstrukcja grafu poprzez obliczanie macierzy podobieństwa nie jest oczywiście jedynym zadaniem, które może służyć do oceny skuteczności algorytmów generujących zanurzenia wierzchołków. Innym ważnym obszarem badań może być zastosowanie stworzonych zanurzeń w uczeniu maszynowym, na przykład w zadaniach klasyfikacji wierzchołków. 

    Kolejną wartą uwagi kwestią jest możliwość połączenia techniki szkicowania znanej z algorytmu \texttt{FastExpSketch} ze schematem działania \texttt{SGSketch}. Algorytm ten działa na podobnej zasadzie, jak \texttt{NodeSketch}, ale z uwagi na efektywny mechanizm aktualizacji szkiców, może być stosowany do analizy dynamicznych grafów. Zastąpienie wykorzystanego w nim szkicowania przez \texttt{FastExpSketch} mogłoby potencjalnie polepszyć precyzję przy rekonstrukcji grafu.