\chapter{Sformułowanie problemu}

\section{Analiza grafów}

    Analiza danych jest dynamicznie rozwijającą się dziedziną informatyki, znajdującą zastosowania w wielu gałęziach przemysłu i badaniach naukowych. Wielka różnorodność rozważanych zbiorów danych, pochodzących z odmiennych źródeł, indukuje potrzebę znajdowania wszechstronnych i efektywnych struktur danych i algorytmów, które mogą służyć do ich reprezentacji i przetwarzania. Grafy doskonale nadają się jako narzędzie do tego typu zadań ze względu na ich wrodzoną zdolność do modelowania złożonych relacji i struktur, od sieci społecznościowych i topologii Internetu po systemy biologiczne i sieci transportowe. Dzięki tej wszechstronności algorytmy grafowe znajdują dziś zastosowania w praktyce, napędzając innowacje i wspomagając przetwarzanie coraz bardziej obszernych zestawów informacji. Pomimo że grafy towarzyszą informatyce niemal od samych jej początków, to jednak rozwój tej dziedziny nie ustaje, zwłaszcza że ilość i złożoność danych stale rośnie. W dzisiejszej erze, w której rozmiary danych często osiągają ogromne rozmiary, istnieje potrzeba dostosowania metodologii opartych na grafach do bardziej efektywnego przetwarzania informacji. 

    W niniejszej pracy będziemy posługiwać się głównie pojęciem grafu prostego, określanego po prostu jako graf. Będziemy go oznaczać przez $G = (V,E)$ - graf, gdzie $V$ - zbiór wierzchołków i $E \subseteq V \times V$ - zbiór krawędzi. W domyśle będziemy skupiać się na grafach ważonych.

\section{Główne sposoby modelowania problemu}
    W niniejszej pracy pochylamy się nad kwestią analizy wielkich zbiorów danych, przedstawionych w postaci grafów. Jednak przed przystąpieniem do omawiania istniejących lub konstrukcji nowych rozwiązań, należy zastanowić się nad istotą problemu, z którym się mierzymy oraz wymaganiami i ograniczeniami, które proponowane algorytmy powinny spełniać. Kluczową kwestią jest więc wybór sposobu modelowania problemu. W kontekście analizy grafów możemy wyróżnić kilka ważnych i użytecznych modeli.

    \subsection{Model klasyczny}
        W tradycyjnej analizie grafów przyjmuje się dość prosty model, gdzie cały graf reprezentujący zbiór danych jest nam dany na wejściu do algorytmu. W praktycznych zastosowaniach jest on zazwyczaj reprezentowany przez macierz sąsiedztwa lub listę sąsiedztwa, choć istnieją również alternatywne reprezentacje, jak macierz incydencji \cite{Wilson_2015}. Charakterystyczną cechą tego modelu, odróżniającą go od omawianych dalej, jest fakt, że dostępna wiedza o grafie jest pełna i dostępna w dowolnym momencie działania algorytmu. Zazwyczaj zakładamy również, że jest on niezmienny w czasie. Jest on niewątpliwie najprostszym i jednocześnie potężnym modelem, stąd też przez dekady to na nim opierały się badania w zakresie analizy grafów. Jednak zapamiętanie całego grafu wiąże się ze sporym narzutem pamięciowym, dla macierzy i listy sąsiedztwa odpowiednio rzędu $O(|V|^2)$ i $O(|E|)$. W świecie ogromnych grafów, gdzie rozmiary analizowanych zbiorów krawędzi mogą sięgać rzędu miliardów, taka złożoność może być nieakceptowalna. 

    \subsection{Strumień grafowy}
        W odpowiedzi na charakterystykę problemu przetwarzania ogromnych zbiorów danych powstał model strumieniowy. Graf jest w nim reprezentowany przez strumień krawędzi, napływających stopniowo. Zakładamy, że zapisanie tego grafu w klasyczny sposób jest niepraktyczne lub niemożliwe ze względu na ograniczoną pamięć. Algorytmy oparte na tym modelu powinny więc działać on-line, na bieżąco aktualizując swój stan i będąc gotowe na obsługę zapytań w dowolnym momencie. Z uwagi na rozmiar, dynamikę i często nieznaną charakterystykę danych, takie metody muszą często pomijać niektóre, mniej istotne w danym kontekście informacje o grafie, ograniczając się do tych kluczowych. W związku z tym często dopuszcza się przybliżone odpowiedzi na zapytania, jednak najlepiej z rozsądnym ograniczeniem na możliwy błąd. 

        Dodatkowo możemy wyróżnić kilka podkategorii \textcolor{red}{w ramach strumieni danych grafowych}. Jeden z najważniejszych podziałów dotyczy, tego, jakiego typu zmiany mogą zachodzić w strukturze grafu. Z uwagi na tę kwestię będziemy wyróżniać dwa typy grafów. Graf nazwiemy statycznym, jeśli do grafu krawędzie są jedynie dodawane i raz ustanowione, nigdy nie znikną. W uproszczeniu możemy założyć, że graf, który badamy, jest stały i niezmienny, ale o kolejnych krawędziach dowiadujemy się stopniowo, gdy pojawiają się one w strumieniu. Z kolei grafy dynamiczne to takie, które dopuszczają szerszą gamę operacji, przede wszystkim usuwanie wcześniej istniejących krawędzi. Może to być przydatne przy reprezentowaniu szybko zmieniających się zbiorów danych, takich jak np. informacje o ruchu samochodowym czy podejrzanych aktywnościach na kontach bankowych. Strumienie grafowe będą głównym modelem rozważanym w ramach niniejszej pracy. 

        \subsubsection*{Definicja formalna}
            Niech $G = (V,E)$ - graf. Strumieniem grafowym nazywamy ciągłą sekwencję elementów, z których każdy ma postać trójki $e_i = (<s_i, d_i>; w_i, t_i)$, gdzie $s_i, d_i$ wierzchołki grafu $G$ i przez parę $<s_i, d_i>$ oznaczamy krawędź pomiędzy nimi. Z kolei $w_i$ i $t_i$ to odpowiednio waga tej krawędzi i moment jej wystąpienia. Określona krawędź może powtarzać się w różnych momentach czasowych z różnymi wagami. Zazwyczaj przyjmujemy, że wagi kolejnych wystąpień krawędzi są akumulowane. W literaturze można również spotkać nieco inne definicje, głównie różniące się dokładną postacią strumieniowanej krotki np. dla grafów dynamicznych może przybrać postać czwórki $e_i = (<s_i, d_i>; w_i, t_i, op)$, gdzie $op \in \{+, -\}$ indykuje typ operacji, a więc czy dana krawędź jest dodawana, czy usuwana z grafu\cite{Pacaci_Bonifati_Özsu_2020}.    
        
    \subsection{Model półstrumieniowy}
        Model półstrumieniowy\cite{Feigenbaum_Kannan_McGregor_Suri_Zhang_2005} (ang. \emph{semi-streaming model}) różni się modelu strumieniowego w dwóch głównych kwestiach. Po pierwsze, narzuca on konkretne ograniczenia na pamięć wykorzystywaną przez algorytm, najczęściej $O(|V| polylog(|V|))$, a więc dla gęstych grafów znacznie mniejszą niż rozmiar grafu. Po drugie, wejście może być skanowane wielokrotnie, zwykle stałą lub logarytmiczną liczbę razy. Model ten można uznać więc za rodzaj pomostu między klasyczną analizą grafów, w których dane znane są od początku i nie istnieją ograniczenia na dostęp do nich, a modelem strumieniowym, który nie pozwala na wielokrotne przeglądanie wcześniejszych krawędzi. Model ten jest często wybierany przez badaczy analizujących konkretne, złożone problemy grafowe takie, jak np.  wyznaczanie najkrótszych ścieżek \cite{Elkin_Trehan_2022} lub minimalnego drzewa rozpinającego \cite{Ahn_Guha_McGregor_2012} przy rygorystycznych ograniczeniach pamięciowych. Podobnie jak w przypadku strumieni grafowych, możemy w tym modelu rozważać grafy statyczne i dynamiczne.

    \subsection{Model rozproszony}
        W wielu praktycznych zastosowaniach takich, jak analiza sieci społecznościowych, dane napływają z różnych źródeł -- np. serwerów rozsianych po świecie i obsługujących różne obszary. Kolejne paczki danych są często relatywnie niezależne od siebie i mogą być rozpatrywane oddzielnie. W takich przypadkach wygodnie jest rozważać model rozproszony analizy grafów. W tym modelu dane są dzielone pomiędzy wiele węzłów obliczeniowych. Takie podejście umożliwia przetwarzanie równoległe, skracając czas obliczeń i ograniczając wielkość przesyłanych danych. Większość obliczeń jest wykonywana lokalnie, bez konieczności angażowania jednej centralnej jednostki. Komunikacja między węzłami ogranicza się do niezbędnych w danym przypadku aktualizacji, zamiast obejmować wszystkie dane. Należy pamiętać o wyzwaniach wynikających z często niepełnej wiedzy węzłów, która może utrudniać rozwiązywanie bardziej złożonych problemów. Obszar rozproszonej analizy grafów znalazł szerokie zastosowania w praktyce, czego dobrym przykładem są zaawansowane platformy ułatwiające pracę w tym modelu, takie jak Google Pregel\cite{Malewicz_Austern_Bik_Dehnert_Horn_Leiser_Czajkowski_2010}, czy Apache Spark GraphX\cite{Xin_Gonzalez_Franklin_Stoica_2013}. 