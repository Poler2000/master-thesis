\documentclass{article}
	
\usepackage{xparse}
	\usepackage[margin=2cm]{geometry}
	\usepackage{enumerate} 
    \usepackage{textcomp}
	
    \frenchspacing
	\linespread{1.2}

    \usepackage[polish]{babel}
    \usepackage[utf8]{inputenc}
    \usepackage[ruled,linesnumbered]{algorithm2e}
    \usepackage{caption}
    \usepackage{subcaption}

    \usepackage{polski}

	\usepackage{amsthm}
	\usepackage{amsmath}
	\usepackage{amsfonts}
	\usepackage{float}
	\usepackage{hhline}
	\usepackage{graphicx}
	\usepackage{multirow}
	\usepackage{mathtools}

\title{\textbf{Praca Magisterska \\
    Wykaz i uwagi do literatury}}

\author{Paweł Polerowicz}
\date{Grudzień 2023}

\begin{document}
    \maketitle
    \section{Zasadnicze struktury danych}
        \subsection{gSketch\cite{Zhao_Aggarwal_Wang_2011}}
            \textbf{Autorzy: } P. Zhao, C. C. Aggarwal, and M. Wang\\
            \textbf{Tytuł: } gSketch: On Query Estimation in Graph Streams\\
            \textbf{Rok publikacji: } 2011 \\
            Artykuł ten stanowi jedną z pierwszych prób zastosowania szkiców danych do obsługi prostych zapytań dotyczących grafu, takich jak częstość występowania danej krawędzi w strumieniu lub gęstość wybranego podgrafu. Autorzy proponują wykorzystanie probabilistycznej metody Count Min, opartej o dwuwymiarową tablicę do składowania częstotliwości krawędzi. Odpowiednie komórki są wyznaczane przez funkcje haszujące zastosowane na krawędziach. Algorytm wykorzystuje próbkę testową do podziału zbioru krawędzi na podzbiory, bazując na ich częstotliwości w taki sposób, aby efektywnie wykorzystać dostępną pamięć, a następnie przetwarza właściwy strumień danych. Metoda jest stratna, a praktycznym wyzwaniem jest odpowiednie wyważenie parametrów tak, aby zachować balans między zużytą pamięcią a dokładnością wyników.

        \subsection{TCM\cite{Tang_Chen_Mitra_2016}}
            \textbf{Autorzy: } N. Tang, Q. Chen, and P. Mitra\\
            \textbf{Tytuł: } Graph Stream Summarization.
            From Big Bang to Big Crunch\\
            \textbf{Rok publikacji: } 2016\\
            Autorzy proponują strukturę TCM do przechowywania informacji o strumieniowanym grafie. Ma ona postać macierzy o boku długości $m$, gdzie $m$ jest pewną stałą. W jej komórkach składowane są wagi krawędzi. Rząd i kolumna komórki odpowiadającej danej parze wierzchołków są wyznaczane przez wynik funkcji haszującej. Czas obliczania hasza jest stały, a co za tym idzie, złożoność czasowa zapytań i wstawiania nowych krawędzi również. Teoretyczna złożoność pamięciowa jest natomiast stała i wynosi $O(m^2)$. W praktycznych zastosowaniach wybór m zależy jednak często od liczby krawędzi i przejmuje się $m$ rzędu $O(\sqrt{|V|})$. Dokładność rezultatów może być niska ze względu na kolizje haszy i zależy od rozmiaru macierzy. Użyteczność tej struktury w bazowej formie jest dyskusyjna, stanowi ona jednak punkt wyjściowy dla bardziej zaawansowanych rozwiązań.

        \subsection{gMatrix\cite{Khan_Aggarwal_2016}}
            \textbf{Autorzy: } A. Khan and C. Aggarwal.\\
            \textbf{Tytuł: } Query-friendly compression of graph streams\\
            \textbf{Rok publikacji: } 2016\\
            Autorzy spoglądają w kierunku gSketch, zauważając potencjał tej metody, ale jednocześnie celnie wskazując jej ograniczenia. Przede wszystkim, wymaga ona próbki testowej dobrze reprezentującej właściwości grafu. Jej skuteczne wyznaczenie może być trudne w wielkich grafach o dynamicznej strukturze. Dodatkowo, wspierane są głównie zapytania o częstotliwość występowania krawędzi, brakuje natomiast np. tych dotyczących osiągalności lub efektywnego wyznaczenia wierzchołków o szczególnie wysokiej częstotliwości. W odpowiedzi na te wyzwania zaproponowana zostaje 3-wymiarowa struktura gMatrix, gdzie dwa wymiary odpowiadają haszom wierzchołków, a trzeci numerowi użytej funkcji haszującej. Trzeci wymiar w tym przypadku pozwala zwiększyć dokładność zapytań, co odróżnia tę metodę od TCM. Podobnie jak tam, pamięć jest ograniczona przez wybrane rozmiary macierzy, a złożoność czasowa zależy od liczby użytych funkcji haszujących.  


        \subsection{GSS\cite{Gou_Zou_Zhao_Yang_2019, Gou_Zou_Zhao_Yang_2023}}
            \textbf{Autorzy: } X. Gou, L. Zou, C. Zhao, and T. Yang\\
            \textbf{Tytuł: } Graph Stream Sketch: Summarizing Graph
            Streams with High Speed and Accuracy\\
            \textbf{Rok publikacji: } 2019 (2023)\\

        \subsection{GS4\cite{Ashrafi-Payaman_Kangavari_Hosseini_Fander_2020}}
            \textbf{Autorzy: } N. Ashrafi-Payaman, M. R. Kangavari, S. Hosseini, and A. M. Fander.\\
            \textbf{Tytuł: } GS4: Graph stream summarization
            based on both the structure and semantics\\
            \textbf{Rok publikacji: } 2020\\

        \subsection{Scube\cite{Chen_Zhou_Chen_Jin_2022}}
            \textbf{Autorzy: } M. Chen, R. Zhou, H. Chen, and H. Jin.\\
            \textbf{Tytuł: } Efficient summarization for skewed graph streams\\
            \textbf{Rok publikacji: } 2022\\

        \subsection{Horae\cite{Chen_Zhou_Chen_Xiao_Jin_Li_2022}}
            \textbf{Autorzy: } M. Chen, R. Zhou, H. Chen, J. Xiao, H. Jin, and B. Li.\\
            \textbf{Tytuł: } Horae: A graph stream summarization structure for
            efficient temporal range query\\
            \textbf{Rok publikacji: } 2022\\

        \subsection{Auxo\cite{Jiang_Chen_Jin_2023}}
            \textbf{Autorzy:} Z. Jiang, H. Chen, and H. Jin \\
            \textbf{Tytuł:} Auxo: A scalable and efficient graph stream summarization structure. \\
            \textbf{Rok publikacji:} 2023 \\

            Przedstawiona zostaje struktura Auxo. Jest ona w dużej mierze rozwinięciem idei TCM i GSS. Głównym założeniem nowej struktury jest wprowadzenie wielu macierzy haszy, tworzących drzewo prefiksowe. Podpisy wierzchołków są wbudowywane w strukturę drzewa, co pozwala zredukować zużytą pamięć. Czas aktualizowania i podstawowych zapytań jest logarytmiczny względem liczby krawędzi. Koszt pamięciowy autorzy dość niefortunnie określają jako $O(|E|(1 - log|E|))$. W rzeczywistości maksymalny rozmiar struktury jest ograniczony i zależy od długości użytych podpisów. 

    \section{Najkrótsze ścieżki}
        \subsection{All Pairs SP for undirected graphs\cite{Roditty_Zwick_2012}}
            \textbf{Autorzy: } \\
            \textbf{Tytuł: } \\
            \textbf{Rok publikacji: } \\

        \subsection{QbS\cite{Wang_Wang_Koehler_Lin_2021}}
            \textbf{Autorzy: } \\
            \textbf{Tytuł: } \\
            \textbf{Rok publikacji: } \\

        \subsection{SP2\cite{Dolgorsuren_Xu_Khan_Jeong_Lee_2016}}
            \textbf{Autorzy: } \\
            \textbf{Tytuł: } \\
            \textbf{Rok publikacji: } \\

        \subsection{Road networks\cite{Aggarwal_Gollapudi_Raghavender_Sinop_2021}}
            \textbf{Autorzy: } \\
            \textbf{Tytuł: } \\
            \textbf{Rok publikacji: } \\

        \subsection{$(1+\epsilon)$-approximate shortest paths\cite{Elkin_Trehan_2022}}
            \textbf{Autorzy: } \\
            \textbf{Tytuł: } \\
            \textbf{Rok publikacji: } \\

    \section{Literatura pomocnicza}
        \subsection{Semi-streaming model\cite{Feigenbaum_Kannan_McGregor_Suri_Zhang_2005}}
            \textbf{Autorzy: } \\
            \textbf{Tytuł: } \\
            \textbf{Rok publikacji: } \\

        \subsection{Subgraph Search\cite{Li_Zou_Ozsu_Zhao_2019}}
            \textbf{Autorzy: } \\
            \textbf{Tytuł: } \\
            \textbf{Rok publikacji: } \\

        \subsection{Graph summarization survey\cite{Liu_Safavi_Dighe_Koutra_2018}}
            \textbf{Autorzy: } \\
            \textbf{Tytuł: } \\
            \textbf{Rok publikacji: } \\

        \subsection{Efficient Sketching algorithm\cite{Lemiesz_2023}}
            \textbf{Autorzy: } \\
            \textbf{Tytuł: } \\
            \textbf{Rok publikacji: } \\

    \bibliographystyle{abbrv} 
    \bibliography{bibliography}
\end{document}
